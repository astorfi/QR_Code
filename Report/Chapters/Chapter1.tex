\chapter{Introduction}

QR-codes are widely used today. For instance smartphones and lots of other devises can be used as QR-code scanners and extract data or on the reverse side generate a QR code and secure it by using a private key. It means that regarding the dimension of QR code, it is very hard(almost imposing) to extract the information without having the specific key.\\
By considering the interesting application of quick response codes, investigation of pattern standards, generating and information extraction of them is of great interest.


\section{Project Backgrounds}

The reason for performing this project is to deep understanding of image feature extraction and perspective transformation. Feature extraction and scene understanding is considered in general sense because according to the nature of QR-codes we don't need perfect reconstruction(For example deblurring or noise elimination) of the image. Recognizing the exact place of black and white square(with any size) would be enough. But it is not even a simple task because the image might be corrupted by any definition. \\
Considering the aforementioned restrictions, the pattern recognition plays an important role in QR-code reconstruction which is the second part of this project. Then after perfect reconstruction of the QR-Code matrix, the decoding of the data is the other task which is not related to scene understanding. In decoding part we consider the standards of designing QR-codes to reverse the procedure. 

\section{Considerations}

In general framework, this project is divided to theory and MATLAB simulations. The theory first addresses the Encoding\& Generating standard patterns of QR codes(which related to scene understanding and feature extraction of the image) and then provide fair knowledge about the decoding part. \\
The simulation is consist of taking a picture, then recognizing the pattern, reconstruction of QR-code matrix and finally decode the data set and display the message. In order to consider the algorithm efficiency, the output also consist the run time of the program.


\section{Project Outlines}

Discuss about different chapters ...